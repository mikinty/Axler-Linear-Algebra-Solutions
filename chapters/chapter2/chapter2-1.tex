\section{Span and Linear independence}

Linear combination, span and linear independence (it's in the title duh) are the most definitions in this chapter.

\begin{definition}
  A \textbf{linear combination} of a list $v_1, \dots, v_m$ of vectors in $V$ is a vector of the form
  \begin{equation*}
    \sum_{i=1}^m a_iv_i,
  \end{equation*}
  where $a_1, \dots, a_m \in \mathbf{F}$.
\end{definition}

\begin{definition}
  The set of all linear combinations of a list of vectors $v_1, \dots, v_m$ in $V$ is called the \textbf{span} of the list, denoted
  \begin{equation*}
    \vspan(v_1, \dots, v_m) = \pbrac{
      \sum_{i=1}^m a_iv_i \, :\, a_1, \dots, a_m \in \mathbf{F}.
    }
  \end{equation*}
  The span of the empty list $()$ is defined to be $\pbra{0}$.
\end{definition}

\begin{definition}
  \begin{itemize}
    \item A list $v_1, \dots, v_m$ of vectors in $V$ is called \textbf{linearly independent} if the only choice $a_1, \dots, a_m \in \mathbf{F}$ that makes
          \begin{equation*}
            \sum_{i=1}^m a_iv_i = 0
          \end{equation*}
          is $a_1 = \cdots = a_m = 0$.
    \item The empty list $()$ is also declared to be linearly independent.
  \end{itemize}
\end{definition}

\bx{
  We have
  \begin{align*}
    \vspan(v_4, v_3 - v_4)                                                  & = \vspan(v_4, v_3)           \\
    \vspan(v_4, v_3-v_4, v_2-v_3) = \vspan(v_4, v_3, v_2-v_3)               & = \vspan(v_4, v_3, v_2)      \\
    \vspan(v_4, v_3-v_4, v_2-v_3, v_1-v_2) = \vspan(v_4, v_3, v_2, v_1-v_2) & = \vspan(v_4, v_3, v_2, v_1)
  \end{align*}
  \label{chap2:1:pr1}
}

\bx{
  \ea{
    \item $v=0$ does not work because we have $a_1 \neq 0, a_1 v = 0$. Otherwise, $v \neq 0 \implies av \neq 0$ for $a \neq 0$.
    \item If one is a scalar multiple of another, then you can write $v_1 = kv_2 \implies v_1 - kv_2 = 0$.
    \item Yes, tuple values are in separate coordinates.
    \item Yes.
  }
}

\bx{
  We want $(5, 9, t)$ to be a linear combination of the first two vectors, i.e.
  \begin{equation*}
    x(3, 1, 4) + y(2, -3, 5) = (5, 9, t)
  \end{equation*}
  So we have $t = 4x + 5y$, and if we just need to solve
  \begin{align*}
    3x + 2y & = 5 \\
    x - 3y  & = 9
  \end{align*}
  I believe this is $y = -2, x = 3$.
}

\bx{
  This is basically the same as the previous problem.
}

\bx{
  \ea{
    \item $\forall k \in \R, k(1+i) \neq 1-i$
    \item $-i(1+i) = 1-i$
  }
}

\bx{
  This is very similar to \ref{chap2:1:pr1}.
}

\bx{
  True.
}

\setcounter{exercise}{16}
\bx{
  If $\forall j, p_j(2) = 0$, then either $\exists p_j = 0$, in which case these polynomials are not linearly independent.

  The other case is that $\forall p_j \neq 0$. In this case, we know they all have a zero at 2, which means we can divide all of them by $(x-2)$.

  Now we have $m+1$ polynomials who degree is at most $m-1$. We know that $\mathcal{P}_{m-1}(\mathbf{F})$ has rank $m$, so $m+1$ elements in this space will definitely be linear dependent.
}