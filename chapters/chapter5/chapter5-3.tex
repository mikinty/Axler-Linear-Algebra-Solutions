\section{Eigenspaces and Diagonal Matrices}

\begin{theorem}
  Suppose $V$ is finite-dimensional and $T \in \L(V)$.
  Let $\lambda_1,\dots,\lambda_m$ denote the distinct eigenvalues of $T$. Then the following are equivalent:
  \ea{
    \item $T$ is diagonalizable
    \item $V$ has a basis consisting of eigenvectors of $T$
    \item There exist 1-dimensional subspaces $U_1, \dots, U_n$ of $V$, each invariant under $T$, such that
    \begin{equation}
      V = \bigoplus_{i=1}^n U_i
    \end{equation}
    \item $V = \bigoplus_{i=1}^m E(\lambda_i, T)$
    \item $\dim V = \sum_{i=1}^m \dim E(\lambda_i, T)$
  }
\end{theorem}

\begin{definition}
  Suppose $T \in \L(V)$ and $\lambda \in \F$. The \textbf{eigenspace} of $T$ corresponding to $\lambda$, denoted $E(\lambda, T)$, is defined by
  \begin{equation}
    E(\lambda, T) = \vnull(T - \lambda I).
  \end{equation}
  So $E(\lambda, T)$ is the $\vspan$ of all eigenvectors corresponding to $\lambda$ plus the 0 vector.
\end{definition}

\bx{
  $V = \vrange T$ since $T$ is diagonalizable, and $\vnull T = \pbrac{0}$.
}

\bx{
  I don't think it is always true, we just need a $\vnull T$ that is nontrivial.
}

\bx{
  If we have (a) then (b) follows directly. I think (b) also gives (a) pretty easily because we know $\vnull T, \vrange T$ cannot have common elements besides $0$.

  I think (b) gives us (c) directly from the discussion earlier. I think we still need to show (c) $\implies$ (a) or (b)...
}

\bx{
  \TODO
}

I should do more of these problems :)
