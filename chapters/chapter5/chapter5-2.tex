\section{Eigenvectors and Upper-Triangular Matrices}

\bx{
  \ea{
    \item Not actually sure how to prove this is invertible...\TODO
    \item It's like adding all the parts of $T$ back in.
  }
}

\bx{
  Fundamental Theorem of Algebra.
}

\bx{
  $T^2 - I = (T-I)(T+I) = 0$. If $\lambda \neq -1$, then $T \neq -I$, so $T = I$ in this case.
}

\bx{
  $P(P-I) = 0$ means that $\vnull P = \vrange (P-I)$. Stuck here...
}

\bx{
  The key here is that $(STS^{-1})^n$ expands in a way that cancels out all $S, S^{-1}$ except for the outer ones.
  E.g.
  \begin{equation*}
    \pa{
      STS^{-1}
    }^3 =
    (STS^{-1})
    (STS^{-1})
    (STS^{-1})
    = ST \pa{
      S^{-1}
      S
    }
    T
    \pa{
      S^{-1}
      S
    }
    TS^{-1}
    = ST^3S^{-1}
  \end{equation*}
}

\bx{
  Applying $T^n$ to $U$ will still be invariant in $U$.
}

\bx{
  $T^2 = 9 \implies (T-3)(T+3) = 0$.
}

\bx{
  Not sure if this exists.
}

\bx{
  By FTA we can write $p(T) = \prod_{i} (T - \lambda_i I)$
  So for any $(T-\lambda_i I) v = 0$, this implies $\lambda_i$ is an eigenvalue, which is a zero of $p$.
}

\bx{
  \begin{align*}
    p(T)v
     & = \pa{\sum_{i=0}^n a_iT^i}v          \\
     & = \sum_{i=0}^n a_iT^iv               \\
     & = \sum_{i=0}^n a_i\lambda_i^i v      \\
     & = \pa{\sum_{i=0}^n a_i\lambda_i^i} v \\
     & = p(\lambda)v
  \end{align*}
}
