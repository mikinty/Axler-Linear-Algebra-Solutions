\section{Orthogonal Complements and Minimization Problems}

\begin{definition}
  If $U$ is a subset of $V$, then the \textbf{orthogonal complement} of $U$, denoted $U^\perp$, is the set of all vectors in $V$ that are orthogonal to every vector in $U$:
  \begin{equation}
    U^\perp = \pbrac{
      v \in V \,:\, \bangle{v, u} = 0, \forall u \in U
    }
  \end{equation}
\end{definition}

The following theorem is a good optimization technique used in many fields,
\begin{theorem}
  Suppose $U$ is a finite-dimensional subspace of $V, v\in V$, and $u \in U$. Then
  \begin{equation}
    \norm{
      v - P_Uv
    } \leq
    \norm{
      v -u
    }.
  \end{equation}
  Furthermore, the inequality is an equality iff $u = P_Uv$.
\end{theorem}

\bx{
  We can show this by using $V = U \oplus U^\perp$.
}

\bx{
  We can show this by using $V = U \oplus U^\perp$.
}

\bx{
  Our earlier exercises tell us the $u_i$ span $U$ and $w_1$ span $U^\perp$.

  Now if we apply Gram-Schmidt, the corresponding orthonormal vectors will still span the same spaces.
}

\bx{
  Too lazy to do G-S.
}