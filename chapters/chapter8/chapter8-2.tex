\section{Decomposition of an Operator}

\bx{
  By the definition of the eigenvalue, we know $\exists j$ such that for $v \in V$,
  \begin{equation*}
    (N - \lambda I)^jv = N^jv = 0
  \end{equation*}
  so $N$ is nilpotent.
}

\bx{
  Just from definitions we have $(T-0I) = T$ should be nilpotent, but that's over a complex space.

  Why does this not work in the real space? I'm struggling...I think for $\R^2$ I was not able to find anything...
}

\bx{
  \begin{align*}
    Tv        & = \lambda v          \\
    STvS^{-1} & = S(\lambda v)S^{-1} \\
    STS^{-1}v & = SS^{-1}\lambda v   \\
    STS^{-1}v & = \lambda v
  \end{align*}
}

\bx{
  no idea \TODO
}

\bx{
  If $V$ has a basis of eigenvectors of $T$, then every generalized eigenvector is an eigenvector of $T$ by definition.

  If every generalized eigenvector of $T$ is an eigenvector as well, then since we know $V = \bigoplus_i G(\lambda_i, T)$, then it must be the case that $V$ has a basis consisting of eigenvectors of $T$.
}

\bx{
  Too lazy, but just use the proof outline in the book \TODO
}

\bx{
  We can use the same proof in the textbook for square roots, except use the Taylor expansion of $(1+x)^{1/3}$ for inspiration.
}

\bx{
  rip didn't understand this either, related to problem 4 I believe
}

\bx{
  This is just straight bookkeeping and computation.
}