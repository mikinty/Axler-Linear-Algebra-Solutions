\section{Duality}

Duality has been hard for me to understand, but it seems to be some corresponding linear operator that can be constructed from any basis of a vector space.

\begin{definition}
  The \textbf{dual space} of $V$, denoted $V'$, is the vector space of all linear functionals on $V$.
  In other words, $V' = \L(V, \F)$.
\end{definition}

\begin{definition}
  For $U \subset V$, the \textbf{annihilator} of $U$, denoted $U^0$, is defined by
  \begin{equation}
    U^0 = \pbrac{
      \phi \in V' \, :\, \phi(u) = 0, \forall u \in U
    }
  \end{equation}
\end{definition}

Yeah I tried reading through the section and 80\% made sense, but I would still myself rekt by not really understanding what a dual it and how to use it.

At least the matrix definitions are pretty trivial.

If you want to prove the row and column rank are the same, you can use Gaussian elimination instead, although we haven't learned that in this book yet.
But most intro courses would definitely introduce Gaussian elimination, since it's an easy way to solve linear equations.

Too lazy to do exercises, hope I can get by this course without them...