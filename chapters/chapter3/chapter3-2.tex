\section{Null Spaces and Ranges}

\begin{definition}
  The \textbf{null space} is the subset of inputs that gets mapped to zero.

  More formally, $T \in \mathcal{L}(V, W)$,
  \begin{equation}
    \vnull T = \pbrac{
      v \in V \,:\, Tv = 0
    }
  \end{equation}
\end{definition}

We start learning about relationships between dimension, linearity, null spaces, ranges and domains.

\begin{theorem}
  Fundamental Theorem of Linear Maps.
  Suppose $V$ is finite dimensional and $T \in \L(V, W)$. Then $\vrange T$ is finite-dimensional and
  \begin{equation}
    \dim V = \dim\vnull T + \dim\vrange T
  \end{equation}
\end{theorem}

\bx{
  $T(v_1, v_2, v_3, v_4, v_5) = (v_1, v_2, 0, 0, 0)$.
}

\bx{
  \begin{align*}
    (ST)^2 v & = (ST)(ST)v                    \\
             & = (ST)w \tag{$w \in \vnull T$} \\
             & = S(Tw)                        \\
             & = S(0)                         \\
             & = 0
  \end{align*}
}

\bx{
  \ea{
    \item $\dim\vnull T = 0$
    \item $z_i \neq 0$ for some set of $\forall i v_i$ are linearly independent
  }
}

\bx{
  $\dim \R^5 = 5 \neq \dim\null T + \dim\vrange T = 2 + 2$.
}

\bx{
  Let
  \begin{equation*}
    T(w, x, y, z) = (0, 0, w, x)
  \end{equation*}
  Then
  \begin{align*}
    \vrange T & = (0, 0, v_1, v_2) \\
    \vnull T  & = (0, 0, v_1, v_2) \\
  \end{align*}
}

\bx{
  $\dim\vrange T \neq \dim \vnull T$ since they have to add up to 5, so they cannot be equal.
}

gonna power through rest of the book