\section{Products and Quotients of Vector Spaces}

Quotients are something we don't usually get in our intro linear algebra class, so that's cool.

\bx{
  If $T$ is a linear map, then we get that the graph of $T$ is a subspace pretty easily with linear properties.

  If the graph of $T$ is a subspace, then we have that
  \begin{align*}
    (v_1, Tv_1) + (v_2, Tv_2) = (v_1 + v_2, Tv_1 + Tv_2) \in \text{graph of } T
  \end{align*}
  here we see that $Tv_1 + Tv_2 = Tv'$ for some $v' \in V$.
  Then we must have $(v', Tv')$ in the graph as well, which means
  \begin{equation*}
    (v_1+v_2, Tv_1 + Tv_2) = (v', Tv')
  \end{equation*}
  and we have $v' = v_1+v_2$, so we conclude that $Tv_1 + Tv_2 = T(v_1+v_2)$.

  The proof for scalar is pretty similar.
}

\bx{
  If $\exists V_j$ infinite-dimensional, then the cross product could be infinite dimensional.
}

\bx{
  Let $U_1, U_2 = (u_1, u_2, u_3, \dots)$, then
  \begin{equation*}
    U_1 \cross U_2 = \F^\infty = U_1 + U_2,
  \end{equation*}
  but we clearly have multiple choices for $u \in U_1, u' \in U_2$ such that their sum is the same.
}

\bx{
  Just intuition here, one is mapping an $m$ tuple's elements each to $W$, the other is a m-tuple of $V_i \to W$ mappings.
  \label{chap3:5:pr4}
}

\bx{
  Very similar intuition to Exercise \ref{chap3:5:pr4}.
}

\bx{
  Let's take advantage of the last two exercises, so we have
  \begin{align*}
    \L(\F^n, V) & = \prod_{i=1}^n \L(\F, V) \\
                & = \L(\F, V^n) = V^n
  \end{align*}
}

\bx{
  AFSOC $U \neq W$, then WLOG $\exists w \in W, \not\in U$.
  Then we have that
  \begin{equation*}
    v\in V, \forall u \in U \quad v + w \neq v + u
  \end{equation*}
  which violates our assumption.
}

\setcounter{exercise}{13}

\bx{
  \ea{
    \item We can check a few things here
    \begin{itemize}
      \item $0 \in U$
      \item Closed by addition since finite $x_j \neq 0$ will still result in that property
      \item Closed by scalar since the finite $x_j \neq 0$ still holds
    \end{itemize}
    Thus we can conclude the subspace

    \item Consider the mapping $f \,:\, U \to \F^\infty$ where for $u \in U$, say $j$ is the largest $x_j \neq 0$, then
    \begin{equation*}
      f(u) = (\dots, x_j, x_j, x_j, \dots)
    \end{equation*}
    This $f(u)$ has infinitely many $x_i \neq 0$, so $f(u) \not\in u$, but $f(u) \in \F^\infty$.

    Since there are infinite number of coordinates that are nonzero, we must have infinite dimension.
  }
}