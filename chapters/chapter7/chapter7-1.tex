\section{Self-Adjoint and Normal Operators}

Adjoint operators are used a lot in higher-level physics classes, especially in quantum.

\begin{definition}
  Suppose $T \in \L(V, W)$. The \textbf{adjoint} of $T$ is the function $\adj{T}:W\to V$ such that
  \begin{equation}
    \bangle{
      Tv, w
    } = \bangle{
      v, \adj{T}w
    }
  \end{equation}
  for every $v\in V$ an every $w \in W$.
\end{definition}

\begin{theorem}
  If $T \in \L(V, W)$, then $\adj{T} \in \L(W, V)$.
\end{theorem}

\begin{definition}
  An operator on an inner product space is called \textbf{normal} if it commutes with its adjoint.

  So $T \in \L(V)$ is normal if
  \begin{equation}
    T\adj{T} = \adj{T}T.
  \end{equation}
\end{definition}

\bx{
  \begin{align*}
    \bangle{
      (z_1, \dots, z_n),
      \adj{T}(x_1, \dots, x_n)
    }
     & = \bangle{
      T(z_1, \dots, z_n),
      (x_1, \dots, x_n)
    }                            \\
     & = \bangle{
      (0, z_1, \dots, z_{n-1}),
      (x_1, \dots, x_n)
    }                            \\
     & = \sum_{i=2}^n x_iz_{i-1} \\
     & = \bangle{
      (z_1, \dots, z_{n-1}, z_n),
      (x_2, \dots, x_n, 0)
    }
  \end{align*}
  So we see that
  \begin{equation}
    \adj{T}(x_1, \dots, x_n) =
    (x_2, \dots, x_n, 0).
  \end{equation}
}

\bx{
  \begin{align*}
    \bangle{
      \lambda v, v
    }
     & = \bangle{
      Tv, v
    }             \\
     & = \bangle{
      v, \adj{T}v
    }             \\
     & = \bangle{
      v, \conj{\lambda} v
    }
  \end{align*}
}

\bx{
  Let $u \in U, u' \in U^\perp$,
  \begin{equation}
    \bangle{Tu, u'} = 0 \iff \bangle{u, \adj{T}u'} = 0
  \end{equation}
}

\bx{
  skip
}

\bx{
  skip
}

\bx{
  \ea{
    \item We see that
    \begin{align*}
      \bangle{Tp, q} & = p_1x \\
      \bangle{p, Tq} & = q_1x
    \end{align*}
    and these expressions are not equal.

    \item Our basis is not orthonormal.
  }
}

\bx{
  If $ST$ is self-adjoint, then
  \begin{align*}
    ST & = \adj{(ST)} \tag{self-adjoint}           \\
       & = \adj{T}\adj{S} \tag{adjoint properties} \\
       & = TS \tag{$S, T$ self-adjoint}
  \end{align*}

  If $ST = TS$, then we have
  \begin{align*}
    \adj{(ST)} & = \adj{T}\adj{S}               \\
               & = TS \tag{$S, T$ self-adjoint} \\
               & = ST. \tag{by assumption}
  \end{align*}
}

\bx{
  \begin{enumerate}[label=(\roman*)]
    \item $0$ is self-adjoint.
    \item Suppose $S, T$ are self-adjoint. Then
          \begin{equation*}
            \adj{(S+T)} = \adj{S} + \adj{T} = S + T
          \end{equation*}
    \item Suppose $S$ is self-adjoint. Then
          \begin{align*}
            \bangle{\lambda Sv, w}
             & = \lambda \bangle{Sv, w}        \\
             & = \lambda\bangle{v, \adj{S}w}   \\
             & = \bangle{v, \conj{\lambda}S w}
          \end{align*}
          I think $\lambda \in \F$ so we're good here, so $\conj{\lambda} = \lambda$?
  \end{enumerate}
}

\bx{
  Additivity fails because if we try
  \begin{align*}
    (T+S)\adj{(S+T)} & = (T+S)(\adj{S} + \adj{T})                   \\
                     & = T\adj{S} + S\adj{S} + S\adj{T} + T\adj{T},
  \end{align*}
  we're quickly stuck because $T$ and $S$ have no guarantees between them about commutativity.
}