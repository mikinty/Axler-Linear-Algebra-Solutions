\section{Positive Operators and Isometrics}

\begin{definition}
  An operator $T \in \L(V)$ is called \textbf{positive} if $T$ is self-adjoint and
  \begin{equation}
    \forall v\in V \quad
    \bangle{
      Tv, v
    } \geq 0
  \end{equation}
\end{definition}

\begin{definition}
  An operator $S \in \L(V)$ is called an \textbf{isometry} if
  \begin{equation}
    \norm{Sv} = \norm{v}, \forall v \in V.
  \end{equation}
  An operator is an isometry if it preserves norms.
\end{definition}

\bx{
  Consider some $v \in V$. We can represent $v$ as part of the orthonormal basis
  \begin{equation*}
    v = \sum_{i=1}^n a_i e_i
  \end{equation*}
  and now if we consider the positive operator definition
  \begin{align*}
    \bangle{
      Tv, v
    } & = \sum_{i=1}^n \bangle{
      T(a_ie_i), v
    }                                              \\
      & = \sum_{i=1}^n \bangle{
      T(a_ie_i), a_ie_i
    } \tag{Since $\bangle{e_i, e_j} = 0, i\neq j$} \\
      & = \sum_{i=1}^n a_i^2 \bangle{
      T(e_i), e_i
    } \geq 0.
  \end{align*}
  So we can conclude that $T$ is a positive operator.
}

\setcounter{exercise}{2}

\bx{
  If $U$ is invariant under $T\vert_U$, then $T\vert_U u \in U$.

  Now, we have
  \begin{equation*}
    \bangle{T\vert_U u, u} \geq 0
  \end{equation*}
  since $u \in U \subset V \implies u \in V$, and $T$ is a positive operator on $V$.

  This solution seems kinda fishy since I didn't really use the invariant property.
  Perhaps it's just to say that $T$ only operates within $U$?
  We already know that any element $v \in V$ satisfies the positive operator condition, and every element of $U$ is in $V$...
}

\bx{
  For $v \in V$,
  \begin{align*}
    \bangle{
      \adj{T}Tv, v
    } & = \bangle{
      Tv, Tv
    } \geq 0
  \end{align*}

  For $w \in W$,
  \begin{align*}
    \bangle{
      T\adj{T}w, w
    } & = \bangle{
      \adj{T}v, \adj{T}v
    } \geq 0
  \end{align*}
}

\bx{
  Suppose we have $S, T$ positive operators on $V$. Then for $v \in V$,
  \begin{align*}
    \bangle{(S+T)v, v}
     & = \bangle{
      Sv + Tv, v
    }             \\
     & = \bangle{
      Sv, v
    } + \bangle{
      Tv, v
    } \geq 0
  \end{align*}
}

\bx{
  We proceed by induction.

  Base case $k=1$ is easy since that's just the definition.

  Suppose we have $\bangle{T^kv, v} \geq 0$ for some $k \geq 1$.
  Now, consider $\bangle{T^{k+1}v, v}$.
  We know that $T$ has an orthonormal basis with nonnegative eigenvalues $\lambda_i$.
  Then if we consider
  \begin{equation*}
    \bangle{T^kv, v} = \sum_{i=1}^n \bangle{a_ie_i, e_i}
  \end{equation*}
  then if we multiply by $T$, we have
  \begin{equation*}
    \bangle{T^{k+1}v, v} = \sum_{i=1}^n \bangle{\lambda_i a_ie_i, e_i} = \geq 0
  \end{equation*}
  We have to show that $a_i \geq 0$, but this we can prove with a lemma with the positive eigenvalues.
}