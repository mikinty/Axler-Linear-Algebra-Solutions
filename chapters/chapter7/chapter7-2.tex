\section{The Spectral Theorem}

One of the most lit theorems in Linear Algebra, listen up.

\begin{theorem}
  Suppose $\F = \C$ and $T \in \L(V)$. Then the following are equivalent:
  \ea{
    \item $T$ is normal.
    \item $V$ has an orthonormal basis consisting of eigenvectors of $T$.
    \item $T$ has a diagonal matrix with respect to some orthonormal basis of $V$.
  }
  \label{theorem:complex_spectral}
\end{theorem}

\begin{theorem}
  Suppose $\F = \R$ and $T \in \L(V)$. Then the following are equivalent:
  \ea{
    \item $T$ is self-adjoint.
    \item $V$ has an orthonormal basis consisting of eigenvectors of $T$.
    \item $T$ has a diagonal matrix with respect to some orthonormal basis of $V$.
  }
  \label{theorem:real_spectral}
\end{theorem}

\bx{
  Seems like this is true, the spectral theorem is about orthonormal bases and diagonal matrices. In general we can have $T$ with eigenvectors that span $\R^3$.
}

\bx{
  Do we just factor this to $(T - 2I)(T - 3I) = 0$?
}

\bx{
  If we choose
  \begin{equation*}
    T = \begin{pmatrix}
      2 & 1 & 1 \\
      0 & 2 & 1 \\
      0 & 0 & 3
    \end{pmatrix}
  \end{equation*}
  Then we have the expression equal to
  \begin{equation*}
    T^2 - 5T + 6I = \begin{pmatrix}
      0 & -1 & 1 \\
      0 & 0  & 0 \\
      0 & 0  & 0
    \end{pmatrix}
  \end{equation*}
}

\bx{
  If $T$ is normal, by the Complex Spectral Theorem (\ref{theorem:complex_spectral}), we know these statements are true.
}

\bx{
  This is the Real Spectral Theorem (\ref{theorem:real_spectral}).
}

\bx{
  \TODO
}
