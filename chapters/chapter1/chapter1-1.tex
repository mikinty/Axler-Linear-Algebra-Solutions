\section{$\mathbf{R}^n$ and $\mathbf{C}^n$}

\bx{
  Multiply both sides by $\frac{a - bi}{a-bi}$.

  This is known as \textit{rationalizing the denominator}.
}

\bx{
  I think this is meant to be an arithmetic problem, but we can notice that
  \begin{equation*}
    \frac{-1 + \sqrt{3}i}{2} = \cis \pa{\frac{2\pi}{3}} \implies
    \pa{\frac{-1 + \sqrt{3}i}{2}}^3 = \cis \pa{2\pi} = 1
  \end{equation*}
  which is just 1 because we are at the rightmost part of the unit circle.
}

\bx{
  $i = \cis \pi/2 + 2\pi k$ so if we take the square root we get solutions in the form of
  \begin{equation*}
    \cis \pi/4 + \pi k
  \end{equation*}
}

\bx{
  Apply commutativity to real and imaginary parts.
}

\bx{
  Separate real and imginary parts and use field properties.
}

\bx{
  Separate real and imginary parts and use field properties.
}

\bx{
  skip
}

\bx{
  skip
}

\bx{
  skip
}

\bx{
  $x = (1/2, 6, -7/2, 1/2).$
}

\bx{
  We can equate two pairs of tuple values and show that no such $\lambda$ exists after simplifications.
}

Skip the rest