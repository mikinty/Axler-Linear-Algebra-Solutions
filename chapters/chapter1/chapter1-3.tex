\section{Subspaces}
\setcounter{exercise}{0}

Subspace property here is very important

\begin{definition}
  $U \subset V$ is a \textbf{subspace} of $V$ if $U$ is also a vector space.

  The conditions for a subspace $U \subset V$ are:
  \begin{enumerate}
    \item additive identity
          \begin{equation*}
            0 \in U
          \end{equation*}
    \item closed under addition
          \begin{equation*}
            u, w \in U \implies u + w \in U
          \end{equation*}
    \item closed under scalar multiplication
          \begin{equation*}
            a \in \mathbf{F} \andtext u \in U \implies au \in U
          \end{equation*}
  \end{enumerate}
\end{definition}

\bx{
  \ea{
    \item yes
    \item no scalar fails
    \item Not closed under addition
    \item Yes
  }
}

\bx{
  We have to confirm, I'll just point out the ones that don't work.
  \ea{
    \item if $b \neq 0$ then scaling doesn't work
    \item yes
    \item yes
    \item not close dunder addition if $b \neq 0$.
    \item Limit 0 is key, another limit point and this won't work.
  }
}

\setcounter{exercise}{14}

\bx{
  $U + U = U$, since $\forall u \in U, 2u \in U$
}

\bx{
  Commutativity should hold.
}

\bx{
  Yes should be associative via field properties.
}

\bx{
  I think they all should, since $0$ is in the subspace and scalar is closed, so we have $v, -v$ for any $v \in V$.
}

\bx{
  False, $U_1 = (u, 0), U_2 = (0, u), W = (x, y)$
}

\bx{
  $W = (0, w_1, 0, w_2)$
}

\bx{
  Can we not just choose a trivial $W$ like $W = \mathbf{F}^5$?

  Otherwise, we notice that $x, y$ are the only independent variables in the tuple, so we need 3 more tuples, so we just choose $W = (0, 0, x, y, z)$.
  \label{chap1:3:pr21}
}

\bx{
  As we saw in \ref{chap1:3:pr21}, we have 3 degrees of freedom, so we can choose
  \begin{align*}
    W_1 & = (0, 0, w_1, 0, 0) \\
    W_2 & = (0, 0, 0, w_2, 0) \\
    W_3 & = (0, 0, 0, 0, w_3)
  \end{align*}
}

\bx{
  False.

  The intuition here is that the part $W$ is ``missing'' from $V$ is constant, but the trick is that we can have one of the $U_1$ ``over contribute'' and the other ``under contribute''.

  So we can choose like $W = (x, y)$ and then $U_1 = (x, 0)$ and then $U_2 = (x, y)$ and the result is both sums are $\mathbb{R}^2$.
}

\bx{
  skip
}