\section{Determinant}

\begin{definition}
  Suppose $T \in \L(V)$.
  \begin{itemize}
    \item If $\F = \C$, then the \textbf{determinant} of $T$ is the product of the eigenvalues of $T$, with each eigenvalue repeated according to its multiplicity.
    \item If $\F = \R$, then the \textbf{determinant} of $T$ is the product of the eigenvalues of $T_{\C}$, with each eigenvalue repeated according to its multiplicity.
  \end{itemize}
  The determinant of $T$ is denoted by $\vdet T$.

  We later find out to calculate the determinant, for
  \begin{equation}
    A = \begin{pmatrix}
      A_{1, 1} & \cdots A_{1, n}          \\
      \vdots   &                 & \vdots \\
      A_{n, 1} & \cdots A_{n, n}
    \end{pmatrix},
  \end{equation}
  we calculate
  \begin{equation}
    \vdet A = \sum_{(m_1, \dots, m_n) \in \text{perm }n} \pa{
      \text{sign}\pa{
        m_1, \dots, m_n
      }
    } A_{m_1, 1} \cdots A_{m_n, n}.
  \end{equation}
\end{definition}

Most of the proofs in this chapter are pretty bash-y, meaning we just have to explicitly calculate sums and products of matrix entries.

The ending stuff about geometries and polar coordinates and spherical coordinates felt pretty rushed, and was a weird addition to a mostly pure-math book.
I think those topics suit better in a course with more physics or geometry applications.

Sorry...not going to do these exercises.

Was overall a fun read though :)
